\documentclass[fleqn,oneside,12pt]{article}
\date{}
\newcommand{\up}[1]{a_{#1}^{\dagger}}
\newcommand{\down}[1]{a_{#1}}
\newcommand{\ud}{\mathrm{d}}
\newcommand{\EE}{\mathrm{E}}
\newcommand{\diff}[2]{\frac{\ud {#1}}{\ud {#2}}}
\newcommand{\pdiff}[2]{\frac{\partial #1}{\partial #2}}
\newcommand{\fl}{\mathrm{fl}}
\newcommand{\sqrtfrac}[2]{\sqrt{\frac{#1}{#2}}}
\newcommand{\ui}{\mathbf{i}}
\newcommand{\uj}{\mathbf{j}}
\newcommand{\ua}{\mathbf{a}}
\newcommand{\ub}{\mathbf{b}}
\newcommand{\bi}{\bar{i}}
\newcommand{\bj}{\bar{j}}
\newcommand{\ba}{\bar{a}}
\newcommand{\bb}{\bar{b}}
\newcommand{\tb}{\phantom{}^{Q} \hat b}
\newcommand{\erf}{\mathrm{erf}}
\newcommand{\erfc}{\mathrm{erfc}}
\setlength{\parskip}{1ex plus 0.5ex minus 0.2ex}
\setlength{\parindent}{0pt}
\usepackage[top=1in, bottom=1in, left=1in, right=1in]{geometry}
\usepackage{amsmath}
\usepackage[super]{cite}
\usepackage{amssymb}
\usepackage[usenames,dvipsnames]{color}
\usepackage{verbatim}
\usepackage{fancyhdr}
\usepackage{graphicx}
\pagestyle{fancyplain}
\begin{document}
\lhead{Generalized Ultrafast Pump-Probe Scattering}
\rhead{MtzGroup AIMSProp Team}

\section{Development}

\textbf{Literature:} 

\textbf{Geometry:} Consider an ultrafast pump-probe scattering experiment (See
Figure \ref{fig:schem} for an example and definitions of lab-frame coordinates).
For today, we will take the pump to be an optical ultrafast pulse. This pump
strikes an isotropic sample (e.g., provided by a gas jet) at $t=0$, and promotes
a (hopefully) single-photon transition in some of the sample molecules from the
stationary Frank-Condon state $S_{0}$ to an excited state $S_{N}$, setting up
non-adiabatic dynamics. Single-photon absorption is not isotropic, but
proportional to the square of the $\hat z$ component of the transition dipole
moment at the Frank-Condon point. At time delay $t$, the scattering probe (x-ray
or electrons, depending on the experiment) strikes the sample, and takes a
snapshot of the current geometry of the sample in terms of the scattering
pattern on the detector.  Typically, this signal is adjusted by subtracting the
$t=-\infty$ signal, exposing changes in the geometry for the pumped molecule,
and providing a ``molecular movie'' if a number of pump-probe time delays are
sampled. Note that the subtraction of the $t=-\infty$ component of the signal
serves to remove the scattering signal from the non-excited molecules (which
remain in the time-independent steady-state FC wavepacket).

\begin{figure}[h!]
\begin{center}
\includegraphics[width=11cm]{figures/schem}
\end{center}
\caption{An illustration of ultrafast X-ray diffraction (along with UED, this is
one of the main types of pump-probe scattering experiments that we consider).
The X-ray probe arrives a delay of $t$ with respect to the optical pump. In the
perpendicular experimental arrangement (shown here), the polarization vector
$\vec E \propto \hat z$ of the optical pump pulse is perpendicular to the
incident wave vector $\vec k_0 \equiv k \hat y$ of the X-ray probe pulse. The
incident X-ray photons scatter at angle $\theta$ with respect to $\vec k_0$, and
azimuthal angle $\eta$ with respect to $\vec E$. These two angles fully describe
the scattering in the elastic regime, as $|\vec k_0|$ = $|\vec k_s|$. The
scattering vector $\vec s$ is the difference between the incident and scattered
wave vectors $\vec s \equiv \vec k_0 - \vec k_s$. The amplitude of the
scattering vector is $s \equiv |\vec s| = 2 k \sin \theta / 2 = (4 \pi /
\lambda) \sin \theta / 2$. Figure adapted to our notation from Kent Wilson's
excellent 1998 ultrafast scattering paper in JPCA [Jianshu Cao and Kent R.
Wilson, \emph{J. Phys. Chem. A} \textbf{102}, 9523 (1998)].}
\label{fig:schem}
\end{figure}

\textbf{Objective:} As theorists, our objective is to perform a computational
simulation of the non-adiabatic dynamics and time-resolved scattering signal
observable of the ultrafast pump-probe experiment. To make things easy, let us
assume that we have performed the non-adiabatic dynamics simulations, and have a
representation for the density of the wavepacket in terms of a set of weighted
trajectories $\{ r_{A}^{I} (t), w_{I} (t) \}$. Each density-matrix trajectory
basis function (e.g., a pair of AIMS TBFs in the Mulliken or saddle-point
approximation, or an adiabatic trajectory) has a set of time-dependent nuclear
coordinates $r_{A}^{I} (t)$, a time-dependent weight factor $w_{A} (t)$, which
accounts for the weight of the starting configuration in the Franck-Condon point
(Wigner sampling provides even weights, but there may also be weights due to
sampling from different conformers), the overall oscillator strength, and any
changes in weights due to adiabatic state population changes over the dynamics,
and an adiabatic electronic label often denoted as $|I\rangle$.

We also apply one additional convention (not usually seen in AIMS) to our
trajectories: we assert that the transition dipole moment of the $t=0$ frame is
aligned to $\hat z$, and that all subsequent frames follow this lab frame
alignment. This will allow for the efficient treatment of anisotropic absorption
later, by ``reusing'' each trajectory many times with different orientations.
Note that this convention can be applied \emph{after} running AIMS, by rotating
all frames of a given trajectory tree by the rotation matrix to bring the $t=0$
frame into the $\hat z$ axis.

In particular, our job is to produce the time-dependent detector pattern $I
(\theta, \eta, t)$, the detector moments $I_{l} (\theta, t)$, and the
corresponding radial distribution functions $I_{l} (r, t)$.

\textbf{Elastic Scattering:} We will assume we are working in the elastic
scattering regime (no momentum transfer to the sample), which implies that
$|\vec k_0| = |\vec k_s|$. This constraint reduces the dimensionality of the
valid scattering space by 1.
\[
\vec s = 
s 
\{
\cos \theta / 2 \sin \eta,
\sin \theta / 2,
\cos \theta / 2 \cos \eta
\}
\]
Here $s = 2 k \sin \theta / 2 = (4 \pi / \lambda) \sin \theta / 2$. Because of
this relationship, $s$ and $\theta$ are used interchangably in discussing the
scattering.

\textbf{IAM:} In the independent atom model (IAM), the scattering for a given
molecular ``frame'' $I$, with known and fixed orientation, is,
\[
I_{I} (\vec s, t)
=   
\left |
\sum_{A}
f_{A} (s)
e^{i \vec s \cdot \vec r_{A}^{I} (t)}
\right |^2
\]
The spherically-isotropic atomic form factors are available for X-ray scattering
and UED from considerations of the atomic electronic density (see below), or may
be computed from scattering codes like ELSEPA (specialized for UED).

The total scattering signal is computed as,
\[
I (\vec s, t)
=
\sum_{I}
w_{I} (t)
\int_{\mathrm{SO(3)}}
\mathrm{d} \hat R \
\hat R_{zz}^2
\left |
\sum_{A}
f_{A} (s)
e^{i \vec s \cdot \hat R \vec r_{A}^{I} (t)}
\right |^2
/ 
\int_{\mathrm{SO(3)}}
\mathrm{d} \hat R \
\]
Note the ``smearing'' over all sample orientations $\hat R$, with $\hat
R_{zz}^2$ weight for the anisotropic absorption.

Note that,
\[
\int_{\mathrm{SO(3)}}
\mathrm{d} \hat R \
\hat R_{zz}^2
/ 
\int_{\mathrm{SO(3)}}
\mathrm{d} \hat R \
=
\frac{1}{3}
\]

\textbf{Detector Signals:} The detector signal can be expanded as,
\[
I (\vec s, t)
=
\sum_{I}
w_{I} (t)
\sum_{A, B}
f_{A} (s)
f_{B} (s)
K_{AB} (\vec s, t)
\]
where,
\[
K_{AB} (\vec s, t)
=
\int_{\mathrm{SO(3)}}
\mathrm{d} \hat R \
\hat R_{zz}^2
e^{i \vec s^{\dagger} \hat R \vec r_{AB}^{I} (t)}
/
\int_{\mathrm{SO(3)}}
\mathrm{d} \hat R \
\]
The rotation group integrals are worked out in A. H. Zewail and J.S. Baskin,
\textit{Chem. Phys. Chem.}, \textbf{6}, 2261 (2015). For the perpendicular case
(selection weight of $\hat R_{zz}^2$):
\[
K^{\perp} (\vec s)
=
\frac{J_{1} (s r)}{sr}
-
\left [
\sin^2 \gamma
+
(
2 
-
3 
\sin^2 \gamma
)
\cos^2 (\theta / 2)
\cos^2 \eta
\right ]
\frac{J_{2} (s r)}{2}
\]
For the parallel case:
(selection weight of $\hat R_{yy}^2$):
\[
K^{\parallel} (\vec s)
=
\frac{J_{1} (s r)}{sr}
-
\left [
\sin^2 \gamma
+
(
2 
-
3 
\sin^2 \gamma
)
\sin^2 (\theta / 2)
\right ]
\frac{J_{2} (s r)}{2}
\]
For the isotropic case,
(selection weight of $1$):
\[
K^{\emptyset} (\vec s)
=
J_{0} (sr)
\]
In the above $\gamma = \cos^{-1} (\hat r \cdot \hat z)$ and $r = | \vec r |$.
Note that $\sin^2 \gamma = 1 - (\hat r \cdot \hat z)^2 = 1 - r_{z}^2 / r^2$ =
$(r_{x}^2 + r_{y}^2) / r^2$.

The spherical Bessel functions are,
\[
J_{0} (s)
=
\frac{\sin x}{x}
\
:
1
\]
\[
J_{1} (x)
=
\frac{\sin x}{x^2}
-
\frac{\cos x}{x}
\
:
0
\]
\[
J_{2} (x)
=
\left (
\frac{3}{x^2} - 1
\right )
\frac{\sin x}{x}
-
3
\frac{\cos x}{x^2}
\
:
0
\]
The limits at $x = 0$ are displayed. Also note that $\lim_{x \rightarrow 0} J_{1}
(x) / x = 1 /3$.

\textbf{Modified Diffraction Intensity:} Particularly for UED, the raw
diffraction intensity above is too peaked near the origin. To facilitate
analysis, a ``modified'' diffraction intensity is often defined as,
\[
s M(s)
\equiv
s
\frac{I_{\mathrm{mol}} (s)}{I_{\mathrm{at}} (s)}
\]
Here, the atomic contribution of the scattering is,
\[
I_{\mathrm{at}} (s)
\equiv
\sum_{A} | f_{A} (s) |^2
\]
Note that this is time and structure invariant. The molecular scattering signal
is,
\[
I_{\mathrm{mol}} (s)
\equiv
I (s)
-
F
I_{\mathrm{at}} (s)
\]
Here $F = 1/3$ for perpendicular and parallel transitions, $F=1$ for isotropic
transitions. 

\textbf{Anisotropy:} For a perpendicular pump-probe arrangement, the scattering
signal can be exactly decomposed into,\footnote{Note that everyone seems to be
using Legendre polynomials in the literature. This doesn't make sense to me, and
I have empirical evidence that the scattering signal exactly decomposes onto $1$
and $\cos (2 \eta)$ azimuthal basis functions for $\hat R_{zz}^2$ anisotropy.}
\[
I (s, \eta, t)
=
I_{0} (s, t)
+
I_{2} (s, t) \cos (2 \eta)
\]
This decomposition can be computed from two collocation points as,
\[
I_{0} (s, t)
=
\frac{1}{2}
\left [
I (s, \eta=0, t)
+
I (s, \eta=\pi/2, t)
\right ]
\]
\[
I_{2} (s, t)
=
\frac{1}{2}
\left [
I (s, \eta=0, t)
-
I (s, \eta=\pi/2, t)
\right ]
\]

\textbf{Pair Correlation Functions:} The distance pair correlation is computed
from the UED detector signal as,
\[
M (r, t)
\equiv
\int_{0}^{s_{\mathrm{max}}}
\mathrm{d} s \
s M(s)
\sin (sr)
\exp(-\alpha s^2)
\]
Formally, $s_{\mathrm{max}} \rightarrow \infty$ for an exact sine transform. In
practice, a cutoff must be used due to the outer limits of $s$ values that are
scattered. The cutoff function $\exp(-\alpha s^2)$ is used to damp down noise
from large $s$. Additionally, the detector signal is only available from
$s_{\mathrm{min}}$ and up, so the signal is typically assumed to rise
smoothly\footnote{Linearly?} from $s=0$ to the value at $s_{\mathrm{min}}$.

Typically this analysis is applied to the isotropic difference signal $\Delta
M_{0} (s, t) = M_{0} (s, t) - M_{0} (s, t = -\infty)$.

\textbf{Specific Example I: UED at SLAC:} One of the main experiments we are
trying to support is the UED-III endeavor at SLAC. A representative paper of the
approach and setup is from the UED-I run: J. Yang et al, \emph{Nat. Comm.},
\textbf{7}, 11232 (2016). Another good paper is on the instrument design: S.
Weathersby et al, \emph{Rev. Sci. Instr.}, \textbf{86}, 073702. 

In this experiment, 3.7 MeV electrons are used as the probe ($\lambda = 0.30$ pm
$= 0.0030$ \AA{}). The detector is an Andor iXon ultra 888 electron multiplying
charge-coupled device camera (EMCCD), which is positioned 3.1 m from the sample.
The accessible range of $s$ is $3.5$ \AA{}$^{-1}$ to $12$ \AA{}$^{-1}$,
corresponding to scattering angles $\theta$ of $0.0957^{\circ}$ to
$0.328^{\circ}$, or a detector radius of $s_{z}$ of $0.52$ to $1.78$ cm. Large
$s$ is limited by the size of the detector, small $s$ is limited by the $0.4$ cm
diameter hole drilled in the EMCCD for the electron beam (thought this may be
improved in later runs). Note that this is very small angle scattering, but
permits rather high values of $s$ due to the small deBroglie wavelength of
ultra-relativistic electrons. By contrast, x-rays typically scatter over much
larger angles to reach similar values of $s$. For instance, $1.3$ \AA{}
wavelength electrons reach only $s = 9.7$ \AA{}$^{-1}$ over the entire
scattering sphere.

In analysis, the UED team typically uses a cutoff of $\alpha = 7^{-2}$
\AA{}$^{-2}$ in the sine transform above. 

The entire instrument has a time resolution of roughly $100$ fs, and notably
lacks an in-flight diagnostic clock (e.g., ``timestamping''), though a THz
streaking experiment has recently been developed to provide an offline clock
calibration. 

A Ti:Sapphire laser serves as both pump and probe initiator. Part of the laser
pulse initiates electron release from a copper photocathode (photoelectron
release), which are subsequently accelerated to 3.7 MeV by a massive klystron.
Tilting/tuning the klystron accelerates the back of the electron bunch more than
the front, combating Coulomb spread in the probe electron bunch, but this can
induce problems in determining the time delay. 

\textbf{X-Ray Scattering vs. Electron Diffraction:} For ultrafast x-ray
scattering, the form factors $f_{A} (s)$ are the Fourier transform of the
electronic density of a spherical atom (possibly selected to reflect the density
of the local chemical environment in the true molecule),
\[
f_{A} (s)
\equiv
\int_{\mathbb{R}^3}
\mathrm{d} r_1
\
\rho_{A}^{\mathrm{e}} (\vec r_{1})
e^{i \vec s \cdot \vec r_1}
\]

These form factors are tabulated for many common atoms at
\begin{verbatim}
http://lampx.tugraz.at/~hadley/ss1/crystaldiffraction/
atomicformfactors/formfactors.php
\end{verbatim}
(accessed 01/18/2018). The specific parametrization is,
\[
f_{A} (s)
\equiv
\sum_{i=1}^{4}
a_{i}
\exp
\left (
-b_{i}
\left (
\frac{s}{4\pi}
\right )^2
\right )
+
c
\]
With 9 parameters $a_{1-4}$ $b_{1-4}$, and $c$ per atom type. Units are
\AA{}$^{-1}$.

To switch to ultrafast electron diffraction, one must also account for
scattering off of the nucleus and for a modified Jacobian element,
\[
f_{A}^{\mathrm{UED}}
(s)
\equiv
\frac{1}{s^2}
\left [
Z_{A}
-
f_{A}^{\mathrm{XRAY}}
(q)
\right ]
\]
See M. Ben-Nunn, J. Cao, and K. Wilson, JPCA, 101, 8744 (1997) for details.

\section{Example: Iodine}

\textbf{Geometry:} As a key example, we consider a perpendicular pump-probe UED
experiment wherein iodine molecule $I_{2}$ is excited to a hypothetical bound
state via a parallel transition (transition dipole along the molecular
axis).\footnote{Note that this toy problem is chosen for ease of
setup/representative nature, not for rigorous physics.} The classical trajectory
is,
\[
r (t)
=
2.66
+
0.5 \sin (2 \pi t [\mbox{fs}]/ 600)
[ \mbox{\AA{}} ]
\]
Using the techniques above, we compute the detector moments $I_0 (s, t)$ and
$I_2 (s, t)$ for each of the frames in the trajectory (see Figure
\ref{fig:I02}). We use the accessible range of $s \in [3.5, 12.0]$ \AA{}$^{-1}$
for the UED experiment, and use a de Broglie wavelength of $\lambda = 0.0030$
\AA{}, which corresponds to the 3.7 MeV electrons of the experiment.

\begin{figure}[h!]
\begin{center}
\includegraphics[width=3.0in]{figures/i2/I0.png}
\includegraphics[width=3.0in]{figures/i2/I2.png}
\caption{Detector moments $I_{0} (s, t)$ and $I_{2} (s, t)$ for toy iodine
problem. These moments are differences (as are all plots in this section), i.e.,
$I (s, t) \leftarrow I(s, t) - I(s, t=0)$. These moments are of the flavor $s
M(s, \eta, t)$ (as are all plots in this section).}
\label{fig:I02}
\end{center}
\end{figure}

At this point, we can easily compute the detector patterns from $I(s, \eta, t) =
I_{0} (s, t) + I_{2} (s, t) \cos(2 \eta)$. E.g., an interesting frame is $t=132$
fs, and the decomposition into isotropic/anisotropic parts is also shown. A
complete 3D detector movie over $(s, \eta, t)$ is in \texttt{figs/i2-movie} in
this TeX source dir.

\begin{figure}[h!]
\begin{center}
\includegraphics[width=3.0in]{figures/i2/I033.png}
\includegraphics[width=3.0in]{figures/i2/I033-0.png}
\includegraphics[width=3.0in]{figures/i2/I033-2.png}
\caption{Detector patterns $I (s, \eta)$, $I_{0} (s, \eta)$ and $I_{2} (s,
\eta)$ for toy iodine problem ($t = 132$ fs).}
\label{fig:Idet}
\end{center}
\end{figure}

Starting from the moments, e.g., $I_{0} (s, t)$, we can compute the ``molecular
movie'' $I_{0} (r, t)$ by sine transforming. This is done in Figures
\ref{fig:IR} and \ref{fig:IR2} for our toy iodine problem, using a variety of
$s_{\mathrm{min}}$ and $s_{\mathrm{max}}$ cutoffs, and without/with time
blurring with a 100 fs FWHM Gaussian. The upper right figure in Figure
\ref{fig:IR} shows roughly what can be expected with the current experimental
setup. 

Of particular note, the $s_{\mathrm{min}}$ limit due to the hole for the
electron beam induces large ``fringes'' in the signal. It might be plausible
that pushing $s_{\mathrm{min}}$ down could be just as useful as pushing
$s_{\mathrm{max}}$ up in cleaning up these pictures. 

\begin{figure}[h!]
\begin{center}
\includegraphics[width=3.0in]{figures/i2/R0.png}
\includegraphics[width=3.0in]{figures/i2/R0blur.png}
\includegraphics[width=3.0in]{figures/i2/R0_smin.png}
\includegraphics[width=3.0in]{figures/i2/R0blur_smin.png}
% \includegraphics[width=3.0in]{figures/i2/R0_smax.png}
% \includegraphics[width=3.0in]{figures/i2/R0blur_smax.png}
% \includegraphics[width=3.0in]{figures/i2/R0_smin_smax.png}
% \includegraphics[width=3.0in]{figures/i2/R0blur_smin_smax.png}
\caption{Molecular movies $I_0(r, t)$ for toy iodine problem.  On the left is the
$~1$ fs resolution signal. On the right is the 100 fs FWHM time-blurred system
(Gaussian blurring). On the top, the $s_{\mathrm{min}} = 3.5$ \AA{}$^{-1}$. On
the bottom, $s_{\mathrm{min}} \sim 0.0$ \AA{}$^{-1}$. In all cases,
$s_{\mathrm{max}} = 12.0$ \AA{}$^{-1}$.
}
\label{fig:IR}
\end{center}
\end{figure}

\begin{figure}[h!]
\begin{center}
%\includegraphics[width=3.0in]{figures/i2/R0.png}
%\includegraphics[width=3.0in]{figures/i2/R0blur.png}
%\includegraphics[width=3.0in]{figures/i2/R0_smin.png}
%\includegraphics[width=3.0in]{figures/i2/R0blur_smin.png}
\includegraphics[width=3.0in]{figures/i2/R0_smax.png}
\includegraphics[width=3.0in]{figures/i2/R0blur_smax.png}
\includegraphics[width=3.0in]{figures/i2/R0_smin_smax.png}
\includegraphics[width=3.0in]{figures/i2/R0blur_smin_smax.png}
\caption{Molecular movies $I_0(r, t)$ for toy iodine problem.  On the left is the
$~1$ fs resolution signal. On the right is the 100 fs FWHM time-blurred system
(Gaussian blurring). On the top, the $s_{\mathrm{min}} = 3.5$ \AA{}$^{-1}$. On
the bottom, $s_{\mathrm{min}} \sim 0.0$ \AA{}$^{-1}$. In all cases,
$s_{\mathrm{max}} = 24.0$ \AA{}$^{-1}$. Note that $\alpha = 0$ in the sine
transform in this case.
}
\label{fig:IR2}
\end{center}
\end{figure}

\end{document}
